%% Esempio per lo stile supsi
\documentclass[twoside]{supsistudent} 

% per settare noindent
\setlength{\parindent}{0pt}


% Crea un capitolo senza numerazione che pero` appare nell'indice %
\newcommand{\problemchapter}[1]{%
  \chapter*{#1}%
  \addcontentsline{toc}{chapter}{#1}%
\markboth{#1}{#1}
}

% Numerazione delle appendici secondo norma
\addto\appendix{
\renewcommand{\thesection}{\Alph{chapter}.\arabic{section}}
\renewcommand{\thesubsection}{\thesection.\arabic{subsection}}}

\setcounter{secnumdepth}{5} 	%per avere più livelli nei titoli
\setcounter{tocdepth}{5}		%per avere più livelli nell'indice


\titolo{Progetto Android - Meteo App}
\studente{Alessandro Bianchi \vspace{1em}\\ Maura Clerici \vspace{1em}\\ Zyril Errol Gatchalian}
\relatore{-}
\correlatore{-}
\committente{Vanni Galli}
\corso{Ingegneria Informatica}
\modulo{M02074 - Sviluppo di applicazioni mobile}
\anno{2019}



\begin{document}

\pagenumbering{alph}
\maketitle
\onehalfspacing
\frontmatter


\pagenumbering{roman}
\tableofcontents


\newpage
\mainmatter
\pagenumbering{arabic}
\setcounter{page}{1}

\chapter{Requisiti}
Lo scopo del progetto è lo sviluppo di un'applicazione Android per visualizzare i dati meteo di diverse località, con i seguenti requisiti:
\begin{itemize}
\item Applicazione di tipo List - Detail
\item Possibilità di aggiungere nuove location manualmente (con popup, nuova schermata)
\item Utilizzo del GPS per leggere la posizione corrente e mostrarla in lista
\item Salvataggio delle location inserite dall'utente su database SQLite
\item Controllo periodico (tramite Background Service) delle temperature; invio di notifiche se la temperatura locale scende / sale sopra una certa soglia
\end{itemize}


\chapter{Implementazione}


\section{Applicazione di tipo List-Detail}
\lipsum[13]


\section{Aggiunta di nuove location}
\lipsum[13]

\section{Utilizzo GPS}
\lipsum[13]

\section{Persistenza delle locations}
\lipsum[13]

\section{Controllo periodico delle temperature e notifiche}
\lipsum[13]


\end{document}
