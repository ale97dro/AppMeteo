%% Esempio per lo stile supsi
\documentclass[twoside]{supsistudent} 

% per settare noindent
\setlength{\parindent}{0pt}


% Crea un capitolo senza numerazione che pero` appare nell'indice %
\newcommand{\problemchapter}[1]{%
  \chapter*{#1}%
  \addcontentsline{toc}{chapter}{#1}%
\markboth{#1}{#1}
}

% Numerazione delle appendici secondo norma
\addto\appendix{
\renewcommand{\thesection}{\Alph{chapter}.\arabic{section}}
\renewcommand{\thesubsection}{\thesection.\arabic{subsection}}}

\setcounter{secnumdepth}{5} 	%per avere più livelli nei titoli
\setcounter{tocdepth}{5}		%per avere più livelli nell'indice


\titolo{Progetto Android - Meteo App}
\studente{Alessandro Bianchi \vspace{1em}\\ Maura Clerici \vspace{1em}\\ Zyril Errol Gatchalian}
\relatore{-}
\correlatore{-}
\committente{Vanni Galli}
\corso{Ingegneria Informatica}
\modulo{M02074 - Sviluppo di applicazioni mobile}
\anno{2019}



\begin{document}

\pagenumbering{alph}
\maketitle
\onehalfspacing
\frontmatter


\pagenumbering{roman}
\tableofcontents


\newpage
\mainmatter
\pagenumbering{arabic}
\setcounter{page}{1}

\chapter{Requisiti}
Lo scopo del progetto è lo sviluppo di un'applicazione Android per visualizzare i dati meteo di diverse località, con i seguenti requisiti:
\begin{itemize}
\item Applicazione di tipo List - Detail
\item Possibilità di aggiungere nuove location manualmente (con popup, nuova schermata)
\item Utilizzo del GPS per leggere la posizione corrente e mostrarla in lista
\item Salvataggio delle location inserite dall'utente su database SQLite
\item Controllo periodico (tramite Background Service) delle temperature; invio di notifiche se la temperatura locale scende / sale sopra una certa soglia
\end{itemize}


\chapter{Implementazione}


\section{Applicazione di tipo List-Detail}

Le applicazioni di tipo List-Detail costituiscono un formato molto popolare per quanto riguarda le applicazioni mobile: si compongono di solito di due schermate principali:

\begin{itemize}
\item List: una lista di entries costituita in questo caso dall'elenco delle location di cui l'utente vuole conoscere le informazioni meteo

\item Detail: una schermata di dettaglio di una delle entries della schermata List: nel nostro caso, verranno mostrati Di solito è possibile raggiungerla eseguendo un tap sul nome nella lista della precedente schermata
\end{itemize}


\section{Aggiunta di nuove location}
Grazie alla presenza del bottone "+" presente nella toolbar dell'applicazione, è possibile accedere al dialog per l'inserimento di una nuova location di interesse. Dopo aver inserito il nome della location, è sufficiente premere il bottone "OK" per inserirla nel database.

\section{Utilizzo GPS}
Per l'uso del GPS, l'applicazione fa ricorso alla libreria SmartLocation per gestire in modo semplice e veloce il dispositivo. \`E necessario, prima di poter fare uso del GPS, implementare la richiesta all'utente dei permessi necessari per il suo utilizzo.

Periodicamente, l'app chiede al GPS la posizione attuale: in questo modo, nel caso l'utente si muovesse, la sua posizione viene mantenuta aggiornata. Questo meccanismo è gestito da un listener che aggiorna la posizione dell'utente ogni 5 secondi.

I dati ritornati dal GPS sono corrispondenti a latitudine e longitudine a cui il dispositivo si trova.

\section{Persistenza delle locations}
\lipsum[13]

\section{Controllo periodico delle temperature e notifiche}
\lipsum[13]

\section{Invio richieste HTTP}
Le richieste HTTP vengono mandate ogni minuto per ricevere aggiornamenti sulla temperatura della posizione corrente in modo da inviare una notifica nel caso in cui
la temperatura supera un certo valore. Inoltre le richieste vengono mandati quando l'applicazione richiede (si preme su una citta o posizione corrente) le condizioni
atmosferiche di una delle città presenti nella lista.\\
Per l'invio di richieste http abbiamo dovuto apportare diverse modifiche:
\begin{itemize}
  \item aggiungere i permessi nel manifest
  \item creare le classi apposite alle richieste asincrone
\end{itemize}
Per poter fare la richiesta ci siamo basati maggiormente sulla classe MeteoFetcher che si occupa di:
\begin{itemize}
  \item tenere la chiave per poter fare le richieste (ottenuta da una previa registrazione al sito OpenWeather)
  \item aprire la connessione verso l'url di OpenWeather
  \item costruire il messaggio di richiesta
  \item prelevare i dati di risposta con uno stream di byte
  \item convalidazione dei dati ricevuti
  \item parsare il messaggio json ritornato dal sito
  \item prelevare i dati parsati e metterli in un oggetto model Location termporaneo
\end{itemize}

Questa classe è gestita da un'altra classe chiamata MeteoTask in modo da far partire in background la richiesta affichè l'applicazione
non si blocchi fino alla ricezione della risposta da parte del sito.


\end{document}
